% Options for packages loaded elsewhere
% Options for packages loaded elsewhere
\PassOptionsToPackage{unicode}{hyperref}
\PassOptionsToPackage{hyphens}{url}
\PassOptionsToPackage{dvipsnames,svgnames,x11names}{xcolor}
%
\documentclass[
  11pt,
  letterpaper,
  DIV=11,
  numbers=noendperiod]{scrartcl}
\usepackage{xcolor}
\usepackage{amsmath,amssymb}
\setcounter{secnumdepth}{-\maxdimen} % remove section numbering
\usepackage{iftex}
\ifPDFTeX
  \usepackage[T1]{fontenc}
  \usepackage[utf8]{inputenc}
  \usepackage{textcomp} % provide euro and other symbols
\else % if luatex or xetex
  \usepackage{unicode-math} % this also loads fontspec
  \defaultfontfeatures{Scale=MatchLowercase}
  \defaultfontfeatures[\rmfamily]{Ligatures=TeX,Scale=1}
\fi
\usepackage{lmodern}
\ifPDFTeX\else
  % xetex/luatex font selection
\fi
% Use upquote if available, for straight quotes in verbatim environments
\IfFileExists{upquote.sty}{\usepackage{upquote}}{}
\IfFileExists{microtype.sty}{% use microtype if available
  \usepackage[]{microtype}
  \UseMicrotypeSet[protrusion]{basicmath} % disable protrusion for tt fonts
}{}
\makeatletter
\@ifundefined{KOMAClassName}{% if non-KOMA class
  \IfFileExists{parskip.sty}{%
    \usepackage{parskip}
  }{% else
    \setlength{\parindent}{0pt}
    \setlength{\parskip}{6pt plus 2pt minus 1pt}}
}{% if KOMA class
  \KOMAoptions{parskip=half}}
\makeatother
% Make \paragraph and \subparagraph free-standing
\makeatletter
\ifx\paragraph\undefined\else
  \let\oldparagraph\paragraph
  \renewcommand{\paragraph}{
    \@ifstar
      \xxxParagraphStar
      \xxxParagraphNoStar
  }
  \newcommand{\xxxParagraphStar}[1]{\oldparagraph*{#1}\mbox{}}
  \newcommand{\xxxParagraphNoStar}[1]{\oldparagraph{#1}\mbox{}}
\fi
\ifx\subparagraph\undefined\else
  \let\oldsubparagraph\subparagraph
  \renewcommand{\subparagraph}{
    \@ifstar
      \xxxSubParagraphStar
      \xxxSubParagraphNoStar
  }
  \newcommand{\xxxSubParagraphStar}[1]{\oldsubparagraph*{#1}\mbox{}}
  \newcommand{\xxxSubParagraphNoStar}[1]{\oldsubparagraph{#1}\mbox{}}
\fi
\makeatother

\usepackage{color}
\usepackage{fancyvrb}
\newcommand{\VerbBar}{|}
\newcommand{\VERB}{\Verb[commandchars=\\\{\}]}
\DefineVerbatimEnvironment{Highlighting}{Verbatim}{commandchars=\\\{\}}
% Add ',fontsize=\small' for more characters per line
\usepackage{framed}
\definecolor{shadecolor}{RGB}{241,243,245}
\newenvironment{Shaded}{\begin{snugshade}}{\end{snugshade}}
\newcommand{\AlertTok}[1]{\textcolor[rgb]{0.68,0.00,0.00}{#1}}
\newcommand{\AnnotationTok}[1]{\textcolor[rgb]{0.37,0.37,0.37}{#1}}
\newcommand{\AttributeTok}[1]{\textcolor[rgb]{0.40,0.45,0.13}{#1}}
\newcommand{\BaseNTok}[1]{\textcolor[rgb]{0.68,0.00,0.00}{#1}}
\newcommand{\BuiltInTok}[1]{\textcolor[rgb]{0.00,0.23,0.31}{#1}}
\newcommand{\CharTok}[1]{\textcolor[rgb]{0.13,0.47,0.30}{#1}}
\newcommand{\CommentTok}[1]{\textcolor[rgb]{0.37,0.37,0.37}{#1}}
\newcommand{\CommentVarTok}[1]{\textcolor[rgb]{0.37,0.37,0.37}{\textit{#1}}}
\newcommand{\ConstantTok}[1]{\textcolor[rgb]{0.56,0.35,0.01}{#1}}
\newcommand{\ControlFlowTok}[1]{\textcolor[rgb]{0.00,0.23,0.31}{\textbf{#1}}}
\newcommand{\DataTypeTok}[1]{\textcolor[rgb]{0.68,0.00,0.00}{#1}}
\newcommand{\DecValTok}[1]{\textcolor[rgb]{0.68,0.00,0.00}{#1}}
\newcommand{\DocumentationTok}[1]{\textcolor[rgb]{0.37,0.37,0.37}{\textit{#1}}}
\newcommand{\ErrorTok}[1]{\textcolor[rgb]{0.68,0.00,0.00}{#1}}
\newcommand{\ExtensionTok}[1]{\textcolor[rgb]{0.00,0.23,0.31}{#1}}
\newcommand{\FloatTok}[1]{\textcolor[rgb]{0.68,0.00,0.00}{#1}}
\newcommand{\FunctionTok}[1]{\textcolor[rgb]{0.28,0.35,0.67}{#1}}
\newcommand{\ImportTok}[1]{\textcolor[rgb]{0.00,0.46,0.62}{#1}}
\newcommand{\InformationTok}[1]{\textcolor[rgb]{0.37,0.37,0.37}{#1}}
\newcommand{\KeywordTok}[1]{\textcolor[rgb]{0.00,0.23,0.31}{\textbf{#1}}}
\newcommand{\NormalTok}[1]{\textcolor[rgb]{0.00,0.23,0.31}{#1}}
\newcommand{\OperatorTok}[1]{\textcolor[rgb]{0.37,0.37,0.37}{#1}}
\newcommand{\OtherTok}[1]{\textcolor[rgb]{0.00,0.23,0.31}{#1}}
\newcommand{\PreprocessorTok}[1]{\textcolor[rgb]{0.68,0.00,0.00}{#1}}
\newcommand{\RegionMarkerTok}[1]{\textcolor[rgb]{0.00,0.23,0.31}{#1}}
\newcommand{\SpecialCharTok}[1]{\textcolor[rgb]{0.37,0.37,0.37}{#1}}
\newcommand{\SpecialStringTok}[1]{\textcolor[rgb]{0.13,0.47,0.30}{#1}}
\newcommand{\StringTok}[1]{\textcolor[rgb]{0.13,0.47,0.30}{#1}}
\newcommand{\VariableTok}[1]{\textcolor[rgb]{0.07,0.07,0.07}{#1}}
\newcommand{\VerbatimStringTok}[1]{\textcolor[rgb]{0.13,0.47,0.30}{#1}}
\newcommand{\WarningTok}[1]{\textcolor[rgb]{0.37,0.37,0.37}{\textit{#1}}}

\usepackage{longtable,booktabs,array}
\usepackage{calc} % for calculating minipage widths
% Correct order of tables after \paragraph or \subparagraph
\usepackage{etoolbox}
\makeatletter
\patchcmd\longtable{\par}{\if@noskipsec\mbox{}\fi\par}{}{}
\makeatother
% Allow footnotes in longtable head/foot
\IfFileExists{footnotehyper.sty}{\usepackage{footnotehyper}}{\usepackage{footnote}}
\makesavenoteenv{longtable}
\usepackage{graphicx}
\makeatletter
\newsavebox\pandoc@box
\newcommand*\pandocbounded[1]{% scales image to fit in text height/width
  \sbox\pandoc@box{#1}%
  \Gscale@div\@tempa{\textheight}{\dimexpr\ht\pandoc@box+\dp\pandoc@box\relax}%
  \Gscale@div\@tempb{\linewidth}{\wd\pandoc@box}%
  \ifdim\@tempb\p@<\@tempa\p@\let\@tempa\@tempb\fi% select the smaller of both
  \ifdim\@tempa\p@<\p@\scalebox{\@tempa}{\usebox\pandoc@box}%
  \else\usebox{\pandoc@box}%
  \fi%
}
% Set default figure placement to htbp
\def\fps@figure{htbp}
\makeatother





\setlength{\emergencystretch}{3em} % prevent overfull lines

\providecommand{\tightlist}{%
  \setlength{\itemsep}{0pt}\setlength{\parskip}{0pt}}



 


\KOMAoption{captions}{tableheading}
\makeatletter
\@ifpackageloaded{caption}{}{\usepackage{caption}}
\AtBeginDocument{%
\ifdefined\contentsname
  \renewcommand*\contentsname{Table of contents}
\else
  \newcommand\contentsname{Table of contents}
\fi
\ifdefined\listfigurename
  \renewcommand*\listfigurename{List of Figures}
\else
  \newcommand\listfigurename{List of Figures}
\fi
\ifdefined\listtablename
  \renewcommand*\listtablename{List of Tables}
\else
  \newcommand\listtablename{List of Tables}
\fi
\ifdefined\figurename
  \renewcommand*\figurename{Figure}
\else
  \newcommand\figurename{Figure}
\fi
\ifdefined\tablename
  \renewcommand*\tablename{Table}
\else
  \newcommand\tablename{Table}
\fi
}
\@ifpackageloaded{float}{}{\usepackage{float}}
\floatstyle{ruled}
\@ifundefined{c@chapter}{\newfloat{codelisting}{h}{lop}}{\newfloat{codelisting}{h}{lop}[chapter]}
\floatname{codelisting}{Listing}
\newcommand*\listoflistings{\listof{codelisting}{List of Listings}}
\makeatother
\makeatletter
\makeatother
\makeatletter
\@ifpackageloaded{caption}{}{\usepackage{caption}}
\@ifpackageloaded{subcaption}{}{\usepackage{subcaption}}
\makeatother
\usepackage{bookmark}
\IfFileExists{xurl.sty}{\usepackage{xurl}}{} % add URL line breaks if available
\urlstyle{same}
\hypersetup{
  pdftitle={Trabajo Final Demografía (Versión Preliminar)},
  pdfauthor={Ingrid Dayana Martinez Mendoza; Ana Karen Morelos Bojórquez},
  colorlinks=true,
  linkcolor={blue},
  filecolor={Maroon},
  citecolor={Blue},
  urlcolor={Blue},
  pdfcreator={LaTeX via pandoc}}


\title{Trabajo Final Demografía (Versión Preliminar)}
\usepackage{etoolbox}
\makeatletter
\providecommand{\subtitle}[1]{% add subtitle to \maketitle
  \apptocmd{\@title}{\par {\large #1 \par}}{}{}
}
\makeatother
\subtitle{Informe Técnico: Construcción de Tablas de Mortalidad del
estado de Colima 2010, 2019 y 2021}
\author{Ingrid Dayana Martinez Mendoza \and Ana Karen Morelos Bojórquez}
\date{}
\begin{document}
\maketitle


\section{Introducción}\label{introducciuxf3n}

Las tablas de mortalidad son herramientas fundamental en el análisis
demográfico y actuarial para cuantificar los patrones de mortalidad
poblacional. Este informe describe el procedimiento seguido para la
construcción de las tablas de vida por sexo para el estado de Colima
(2010, 2019 y 2021), documentando las características demográficas
relevantes de la entidad y detallando metodología, fuentes de
información, algoritmos y resultados, con énfasis especial en el impacto
de la pandemia de COVID-19 sobre la mortalidad y la esperanza de vida en
el estatal.

A través de este análisis, se busca proporcionar una base sólida para la
comprensión de los patrones de mortalidad en Colima y su evolución
durante un período que incluye el crítico evento de la pandemia,
ofreciendo percepciones o conociminetos valiosos para la toma de
decisiones en el ámbito de la salud pública y la planificación
actuarial.

\section{Contexto Demográfico del Estado de
Colima}\label{contexto-demogruxe1fico-del-estado-de-colima}

\textbf{Características Poblacionales y su Relación con la Mortalidad}

Colima, siendo uno de los estados con menor extensión territorial en
México, presenta particularidades demográficas que influyen directamente
en su estructura de mortalidad:

\textbf{Estructura Poblacional:}

- Población concentrada estimada de 731,391 habitantes (2020)
distribuidos en 10 municipios, con densidad media-alta

- Urbanización acelerada (85\% en zonas urbanas)

- Transición demográfica acelerada: Proceso de envejecimiento
poblacional acelerado en la última década

- Distribución etaria cambiante: Base infantil reducida con expansión de
adultos mayores

\textbf{Indicadores Demográficos Clave (2020):}

- Tasa de crecimiento poblacional: 1.2\% anual

- Estructura por edad: 0-14 años (25\%), 15-64 años (65\%), 65+ años
(10\%)

- Índice de envejecimiento: 40 adultos mayores por cada 100 niños

- Relación de dependencia: 54 dependientes por cada 100 en edad
productiva

\textbf{Determinantes Sociales y Económicas:}

- Indicadores educativos favorables: superiores al promedio nacional

- Economía especializada basada en actividades portuarias, comerciales y
turísticas

- Cobertura de servicios de salud con disparidades regionales

- Movilidad constante: flujo permanente de personas y mercancías por
actividad portuaria

\textbf{Patrones Epidemiológicos:}

- Transición epidemiológica avanzada con predominio de enfermedades
crónico-degenerativas

- Mortalidad por causas externas relacionadas con actividad portuaria y
turística

- Vulnerabilidad específica ante eventos pandémicos por densidad
poblacional

\section{Diagrama de flujo}\label{diagrama-de-flujo}

\includegraphics[width=0.5\linewidth,height=\textheight,keepaspectratio]{Diagrama_flujo.drawio.png}

\textbf{Descripción del Proceso:}

El diagrama ilustra el flujo completo desde la obtención de datos brutos
del INEGI hasta la generación de tablas de vida y análisis avanzados.
Incluye las etapas de preprocesamiento, cálculo de poblaciones,
procesamiento de defunciones, construcción de tablas de vida, y los
análisis de descomposición y causa-eliminada.

\section{Algoritmos usados}\label{algoritmos-usados}

\textbf{1. Algoritmos de Cálculo de APV}

\begin{itemize}
\tightlist
\item
  \textbf{APV 2010:}
\end{itemize}

\begin{Shaded}
\begin{Highlighting}[]
\NormalTok{N }\OtherTok{\textless{}{-}} \FunctionTok{expo}\NormalTok{(censos\_pro[year}\SpecialCharTok{==}\DecValTok{2010}\NormalTok{]}\SpecialCharTok{$}\NormalTok{pop, }
\NormalTok{          censos\_pro[year}\SpecialCharTok{==}\DecValTok{2020}\NormalTok{]}\SpecialCharTok{$}\NormalTok{pop, }
          \AttributeTok{t\_0 =} \StringTok{"2010{-}06{-}25"}\NormalTok{, }\AttributeTok{t\_T =} \StringTok{"2020{-}03{-}15"}\NormalTok{, }\AttributeTok{t =} \FloatTok{2010.5}\NormalTok{)}
\NormalTok{apv2010 }\OtherTok{\textless{}{-}}\NormalTok{ censos\_pro[year}\SpecialCharTok{==}\DecValTok{2010}\NormalTok{, .(age, sex, N)]}
\NormalTok{apv2010[, year }\SpecialCharTok{:}\ErrorTok{=} \DecValTok{2010}\NormalTok{]}
\end{Highlighting}
\end{Shaded}

\begin{itemize}
\tightlist
\item
  \textbf{APV 2019:}
\end{itemize}

\begin{Shaded}
\begin{Highlighting}[]
\NormalTok{N }\OtherTok{\textless{}{-}} \FunctionTok{expo}\NormalTok{(censos\_pro[year}\SpecialCharTok{==}\DecValTok{2010}\NormalTok{]}\SpecialCharTok{$}\NormalTok{pop, }
\NormalTok{          censos\_pro[year}\SpecialCharTok{==}\DecValTok{2020}\NormalTok{]}\SpecialCharTok{$}\NormalTok{pop, }
          \AttributeTok{t\_0 =} \StringTok{"2010{-}06{-}25"}\NormalTok{, }\AttributeTok{t\_T =} \StringTok{"2020{-}03{-}15"}\NormalTok{, }\AttributeTok{t =} \FloatTok{2019.5}\NormalTok{)}
\NormalTok{apv2019 }\OtherTok{\textless{}{-}}\NormalTok{ censos\_pro[year}\SpecialCharTok{==}\DecValTok{2010}\NormalTok{, .(age, sex, N)]}
\NormalTok{apv2019[, year }\SpecialCharTok{:}\ErrorTok{=} \DecValTok{2019}\NormalTok{]}
\end{Highlighting}
\end{Shaded}

\begin{itemize}
\tightlist
\item
  \textbf{APV 2021:}
\end{itemize}

\begin{Shaded}
\begin{Highlighting}[]
\NormalTok{N }\OtherTok{\textless{}{-}} \FunctionTok{expo}\NormalTok{(censos\_pro[year}\SpecialCharTok{==}\DecValTok{2010}\NormalTok{]}\SpecialCharTok{$}\NormalTok{pop, }
\NormalTok{          censos\_pro[year}\SpecialCharTok{==}\DecValTok{2020}\NormalTok{]}\SpecialCharTok{$}\NormalTok{pop, }
          \AttributeTok{t\_0 =} \StringTok{"2010{-}06{-}25"}\NormalTok{, }\AttributeTok{t\_T =} \StringTok{"2020{-}03{-}15"}\NormalTok{, }\AttributeTok{t =} \FloatTok{2021.5}\NormalTok{)}
\NormalTok{apv2021 }\OtherTok{\textless{}{-}}\NormalTok{ censos\_pro[year}\SpecialCharTok{==}\DecValTok{2020}\NormalTok{, .(age, sex, N)]}
\NormalTok{apv2021[, year }\SpecialCharTok{:}\ErrorTok{=} \DecValTok{2021}\NormalTok{]}
\end{Highlighting}
\end{Shaded}

\textbf{2. Procesamiento de Defunciones}

\begin{itemize}
\tightlist
\item
  \textbf{Promedio para 2010 y datos directos para 2019, 2021:}
\end{itemize}

\begin{Shaded}
\begin{Highlighting}[]
\NormalTok{def\_pro }\OtherTok{\textless{}{-}} \FunctionTok{fread}\NormalTok{(}\StringTok{"data/def\_pro.csv"}\NormalTok{) }\SpecialCharTok{\%\textgreater{}\%}
\NormalTok{  .[year }\SpecialCharTok{\%in\%} \FunctionTok{c}\NormalTok{(}\DecValTok{2009}\NormalTok{, }\DecValTok{2010}\NormalTok{, }\DecValTok{2011}\NormalTok{, }\DecValTok{2019}\NormalTok{, }\DecValTok{2021}\NormalTok{)]}

\NormalTok{def\_pro[, year\_new }\SpecialCharTok{:}\ErrorTok{=} \FunctionTok{ifelse}\NormalTok{(year }\SpecialCharTok{\%in\%} \DecValTok{2009}\SpecialCharTok{:}\DecValTok{2011}\NormalTok{, }\DecValTok{2010}\NormalTok{,}
                            \FunctionTok{ifelse}\NormalTok{(year }\SpecialCharTok{\%in\%} \DecValTok{2018}\SpecialCharTok{:}\DecValTok{2019}\NormalTok{, }\DecValTok{2019}\NormalTok{, year))]}

\NormalTok{def }\OtherTok{\textless{}{-}}\NormalTok{ def\_pro[, .(}\AttributeTok{deaths =} \FunctionTok{mean}\NormalTok{(deaths)), }
\NormalTok{               by }\OtherTok{=}\NormalTok{ .(}\AttributeTok{year =}\NormalTok{ year\_new, sex, age)]}
\end{Highlighting}
\end{Shaded}

\textbf{3. Fórmulas Matemáticas Implementadas}

\begin{itemize}
\item
  \textbf{3.1 Crecimiento Exponencial para APV}

  \begin{itemize}
  \tightlist
  \item
    \textbf{Tasa de crecimiento instantánea:} \[
    r = \frac{\ln(P_T) - \ln(P_0)}{t_T - t_0}
    \]
  \item
    \textbf{Crecimiento exponencial para APV:} \[
    P(t) = P_0 \cdot e^{r \cdot (t - t_0)}
    \]
  \end{itemize}
\item
  \textbf{3.2 Tasas de Mortalidad}

  \begin{itemize}
  \item
    \textbf{Tasa de mortalidad específica:} \[
      m_x = \frac{D_x}{E_x}
    \]
  \item
    \textbf{Tasa central de mortalidad:} \[
    _nm_x = \frac{_nD_x}{_nN_x} \approx \frac{_nd_x}{_nL_x}
    \]
  \item
    \textbf{Fuerza de mortalidad (instantánea):} \[
    \mu(x) = \lim_{n \to 0} nm_x = -\frac{d \ln l(x)}{dx}
    \]
  \end{itemize}
\item
  \textbf{3.3 Conversión mx a qx (Greville-Chiang):} \[
  _nq_x = \frac{n \cdot _nm_x}{1 + (n - _na_x) \cdot _nm_x}
  \]
\item
  \textbf{3.4 Construcción tabla de vida:} \[
  l_0 = 100000
  \] \[
  _nd_x = l_x \cdot _nq_x
  \] \[
  l_{x+n} = l_x - _nd_x
  \] \[
  _nL_x = n \cdot l_{x+n} + _na_x \cdot _nd_x
  \] \[
  T_x = \sum_{y \ge x} L_y
  \] \[
  e_x = \frac{T_x}{l_x}
  \]
\item
  \textbf{3.5 Grupo final abierto:} \[
  \infty L_x = \frac{l_x}{\infty m_x}\]
\item
  \textbf{3.6 Descomposición de Arriaga:}
  \[ \Delta e_0 = \sum_x \left[ \frac{l_x^1}{l_0^1} \cdot \left( \frac{L_x^2}{l_x^2} - \frac{L_x^1}{l_x^1} \right) + \frac{T_{x+1}^2}{l_0^1} \cdot \left( \frac{l_x^1}{l_x^1} - \frac{l_{x+1}^1}{l_x^1} \right) \right] \]
\end{itemize}

\textbf{4. Función Principal de Tablas de Vida}

\begin{Shaded}
\begin{Highlighting}[]
\NormalTok{lt\_abr }\OtherTok{\textless{}{-}} \ControlFlowTok{function}\NormalTok{(x, mx, }\AttributeTok{sex=}\StringTok{"f"}\NormalTok{, }\AttributeTok{IMR=}\ConstantTok{NA}\NormalTok{)\{}
  
\NormalTok{  m }\OtherTok{\textless{}{-}} \FunctionTok{length}\NormalTok{(x)}
\NormalTok{  n }\OtherTok{\textless{}{-}} \FunctionTok{c}\NormalTok{(}\FunctionTok{diff}\NormalTok{(x), }\ConstantTok{NA}\NormalTok{)  }
\NormalTok{  ax }\OtherTok{\textless{}{-}}\NormalTok{ n}\SpecialCharTok{/}\DecValTok{2}    
    
  \CommentTok{\# Ajustes Coale{-}Demeny para edades 0 y 1{-}4}
  \ControlFlowTok{if}\NormalTok{(sex}\SpecialCharTok{==}\StringTok{"m"}\NormalTok{)\{}
    \ControlFlowTok{if}\NormalTok{(mx[}\DecValTok{1}\NormalTok{]}\SpecialCharTok{\textgreater{}=}\FloatTok{0.107}\NormalTok{)\{ }
\NormalTok{      ax[}\DecValTok{1}\NormalTok{] }\OtherTok{\textless{}{-}} \FloatTok{0.330} 
\NormalTok{      ax[}\DecValTok{2}\NormalTok{] }\OtherTok{\textless{}{-}} \FloatTok{1.352} 
\NormalTok{    \} }\ControlFlowTok{else}\NormalTok{ \{}
\NormalTok{      ax[}\DecValTok{1}\NormalTok{] }\OtherTok{\textless{}{-}} \FloatTok{0.045+2.684}\SpecialCharTok{*}\NormalTok{mx[}\DecValTok{1}\NormalTok{]}
\NormalTok{      ax[}\DecValTok{2}\NormalTok{] }\OtherTok{\textless{}{-}} \FloatTok{1.651{-}2.816}\SpecialCharTok{*}\NormalTok{mx[}\DecValTok{1}\NormalTok{]}
\NormalTok{    \} }
\NormalTok{  \} }\ControlFlowTok{else} \ControlFlowTok{if}\NormalTok{(sex}\SpecialCharTok{==}\StringTok{"f"}\NormalTok{)\{}
    \ControlFlowTok{if}\NormalTok{(mx[}\DecValTok{1}\NormalTok{]}\SpecialCharTok{\textgreater{}=}\FloatTok{0.107}\NormalTok{)\{ }
\NormalTok{      ax[}\DecValTok{1}\NormalTok{] }\OtherTok{\textless{}{-}} \FloatTok{0.350} 
\NormalTok{      ax[}\DecValTok{2}\NormalTok{] }\OtherTok{\textless{}{-}} \FloatTok{1.361} 
\NormalTok{    \} }\ControlFlowTok{else}\NormalTok{ \{}
\NormalTok{      ax[}\DecValTok{1}\NormalTok{] }\OtherTok{\textless{}{-}} \FloatTok{0.053+2.800}\SpecialCharTok{*}\NormalTok{mx[}\DecValTok{1}\NormalTok{]}
\NormalTok{      ax[}\DecValTok{2}\NormalTok{] }\OtherTok{\textless{}{-}} \FloatTok{1.522{-}1.518}\SpecialCharTok{*}\NormalTok{mx[}\DecValTok{1}\NormalTok{]}
\NormalTok{    \}  }
\NormalTok{  \}}
  
  \CommentTok{\# Construcción de la tabla de vida}
\NormalTok{  qx }\OtherTok{\textless{}{-}}\NormalTok{ (n}\SpecialCharTok{*}\NormalTok{mx)}\SpecialCharTok{/}\NormalTok{(}\DecValTok{1}\SpecialCharTok{+}\NormalTok{(n}\SpecialCharTok{{-}}\NormalTok{ax)}\SpecialCharTok{*}\NormalTok{mx)}
\NormalTok{  qx[m] }\OtherTok{\textless{}{-}} \DecValTok{1}
\NormalTok{  px }\OtherTok{\textless{}{-}} \DecValTok{1}\SpecialCharTok{{-}}\NormalTok{qx}
\NormalTok{  lx }\OtherTok{\textless{}{-}} \DecValTok{100000} \SpecialCharTok{*} \FunctionTok{cumprod}\NormalTok{(}\FunctionTok{c}\NormalTok{(}\DecValTok{1}\NormalTok{,px[}\SpecialCharTok{{-}}\NormalTok{m]))}
\NormalTok{  dx }\OtherTok{\textless{}{-}} \FunctionTok{c}\NormalTok{(}\SpecialCharTok{{-}}\FunctionTok{diff}\NormalTok{(lx), lx[m])}
\NormalTok{  Lx }\OtherTok{\textless{}{-}}\NormalTok{ n}\SpecialCharTok{*} \FunctionTok{c}\NormalTok{(lx[}\SpecialCharTok{{-}}\DecValTok{1}\NormalTok{], }\DecValTok{0}\NormalTok{) }\SpecialCharTok{+}\NormalTok{ ax}\SpecialCharTok{*}\NormalTok{dx}
\NormalTok{  Lx[m] }\OtherTok{\textless{}{-}}\NormalTok{ lx[m]}\SpecialCharTok{/}\NormalTok{mx[m]}
\NormalTok{  Tx }\OtherTok{\textless{}{-}} \FunctionTok{rev}\NormalTok{(}\FunctionTok{cumsum}\NormalTok{(}\FunctionTok{rev}\NormalTok{(Lx)))}
\NormalTok{  ex }\OtherTok{\textless{}{-}}\NormalTok{ Tx}\SpecialCharTok{/}\NormalTok{lx}
  
  \FunctionTok{return}\NormalTok{(}\FunctionTok{data.table}\NormalTok{(x, n, mx, ax, qx, px, lx, dx, Lx, Tx, ex))}
\NormalTok{\}}
\end{Highlighting}
\end{Shaded}

\section{Codigo usado}\label{codigo-usado}

\begin{itemize}
\item
  \textbf{Fuentes de Información Población:}

  \begin{itemize}
  \item
    Censo de Población y Vivienda 2010:
    \href{https://www.inegi.org.mx/programas/ccpv/2010/}{INEGI 2010}
  \item
    Censo de Población y Vivienda 2020:
    \href{https://www.inegi.org.mx/programas/ccpv/2020/}{INEGI 2020}
  \end{itemize}
\item
  \textbf{Defunciones:}

  \begin{itemize}
  \tightlist
  \item
    Estadísticas Vitales de Mortalidad 2010-2021:
    \href{https://www.inegi.org.mx/programas/mortalidad/}{INEGI EDR}
  \end{itemize}
\end{itemize}

\subsection{Limpieza y preparación de datos
(00\_pre\_procesamiento.R)}\label{limpieza-y-preparaciuxf3n-de-datos-00_pre_procesamiento.r}

Transformar los datos crudos del INEGI en tablas limpias y listas para
análisis.

\textbf{Proceso:}

\begin{enumerate}
\def\labelenumi{\arabic{enumi}.}
\tightlist
\item
  \textbf{Carga de datos originales}

  \begin{itemize}
  \tightlist
  \item
    Lectura de archivos Excel del INEGI con información censal y de
    defunciones
  \item
    Extracción de datos desde rangos específicos de las hojas de cálculo
  \end{itemize}
\item
  \textbf{Limpieza y estandarización}

  \begin{itemize}
  \tightlist
  \item
    \textbf{Edades}: Conversión de formato texto a numérico (``0-4
    años'' → 0)
  \item
    \textbf{Valores numéricos}: Transformación de formatos con comas
    (``1,000'' → 1000)
  \item
    \textbf{Estructura}: Eliminación de filas de totales y encabezados
    redundantes
  \item
    \textbf{Consistencia}: Homogenización de grupos de edad entre
    diferentes años
  \end{itemize}
\item
  \textbf{Tratamiento de valores missing}

  \begin{itemize}
  \tightlist
  \item
    \textbf{Prorrateo inteligente}: Distribución proporcional de valores
    ``No especificado'' en edad y sexo
  \item
    \textbf{Método}: Asignación basada en la estructura conocida de los
    datos
  \item
    \textbf{Objetivo}: Maximizar la información disponible sin
    distorsionar los patrones demográficos
  \end{itemize}
\item
  \textbf{Almacenamiento de datos procesados}

  \begin{itemize}
  \tightlist
  \item
    Exportación de tablas limpias en formato CSV
  \item
    Preservación de la estructura consistente para análisis posteriores
  \end{itemize}
\end{enumerate}

\textbf{Resultado final}: Conjuntos de datos censales y de defunciones
completamente depurados, consistentes metodológicamente y listos para el
cálculo de indicadores demográficos.

\textbf{Tabla Censos pro}

\begin{Shaded}
\begin{Highlighting}[]
\NormalTok{datos }\OtherTok{\textless{}{-}} \FunctionTok{read.csv}\NormalTok{(}\StringTok{"../data/censos\_pro.csv"}\NormalTok{)}
\FunctionTok{head}\NormalTok{(datos, }\DecValTok{10}\NormalTok{)}
\end{Highlighting}
\end{Shaded}

\begin{verbatim}
    X year  sex age       pop
1   1 2010 male   0  5807.782
2   2 2010 male   1 24234.356
3   3 2010 male   5 30708.257
4   4 2010 male  10 31085.859
5   5 2010 male  15 31932.173
6   6 2010 male  20 29850.806
7   7 2010 male  25 26314.710
8   8 2010 male  30 24893.388
9   9 2010 male  35 24162.480
10 10 2010 male  40 20349.003
\end{verbatim}

\textbf{Tabla Defunciones pro}

\begin{Shaded}
\begin{Highlighting}[]
\NormalTok{datos }\OtherTok{\textless{}{-}} \FunctionTok{read.csv}\NormalTok{(}\StringTok{"../data/def\_pro.csv"}\NormalTok{)}
\FunctionTok{head}\NormalTok{(datos, }\DecValTok{10}\NormalTok{)}
\end{Highlighting}
\end{Shaded}

\begin{verbatim}
   year  sex age    deaths
1  1990 male   0 208.41996
2  1990 male   1  68.58290
3  1990 male   5  16.37801
4  1990 male  10  17.40163
5  1990 male  15  39.92139
6  1990 male  20  52.20489
7  1990 male  25  59.16843
8  1990 male  30  45.03952
9  1990 male  35  56.29939
10 1990 male  40  45.03952
\end{verbatim}

\subsection{Calculo de poblaciones a mitad de año
(01\_apv.R)}\label{calculo-de-poblaciones-a-mitad-de-auxf1o-01_apv.r}

Estimar las poblaciones por edad y sexo en puntos intermedios entre los
censos, esencial para el cálculo preciso de tasas de mortalidad.

\textbf{Proceso:}

\begin{enumerate}
\def\labelenumi{\arabic{enumi}.}
\tightlist
\item
  \textbf{Interpolación exponencial de poblaciones}

  \begin{itemize}
  \tightlist
  \item
    \textbf{Método}: Aplicación de crecimiento exponencial entre censos
    2010 y 2020
  \item
    \textbf{Puntos calculados}:

    \begin{itemize}
    \tightlist
    \item
      2010.5 (mitad de año 2010)
    \item
      2019.5 (mitad de año 2019)
    \item
      2021.5 (mitad de año 2021)
    \end{itemize}
  \item
    \textbf{Fórmula}: \(P(t) = P_0 \cdot e^{r \cdot (t - t_0)}\)
  \end{itemize}
\item
  \textbf{Visualización de estructura poblacional}

  \begin{itemize}
  \tightlist
  \item
    \textbf{Pirámides poblacionales}: Gráficas comparativas por edad y
    sexo
  \item
    \textbf{Análisis visual}: Identificación de patrones demográficos y
    cambios temporales
  \item
    \textbf{Escalas adaptadas}: Presentación en millones para mejor
    interpretación
  \end{itemize}
\item
  \textbf{Consolidación de datos}

  \begin{itemize}
  \tightlist
  \item
    \textbf{Unificación}: Integración de APV para los tres años de
    estudio
  \item
    \textbf{Estructura consistente}: Mismo formato para todos los
    períodos
  \item
    \textbf{Validación}: Verificación de coherencia entre estimaciones
  \end{itemize}
\end{enumerate}

\textbf{Resultado final}: Poblaciones estimadas por edad y sexo para
2010, 2019 y 2021, fundamentales para el cálculo robusto de indicadores
de mortalidad.

\textbf{Gráficas de los APV por año}

\begin{figure}

\begin{minipage}{0.50\linewidth}
\pandocbounded{\includegraphics[keepaspectratio]{images/piramide poblacional 2010.jpg}}\end{minipage}%
%
\begin{minipage}{0.50\linewidth}
\pandocbounded{\includegraphics[keepaspectratio]{images/clipboard-1040485046.jpeg}}\end{minipage}%

\end{figure}%

\textbf{Gráfica comparativa}

\includegraphics[width=0.5\linewidth,height=\textheight,keepaspectratio]{images/piramide poblacional comparativa.jpg}

\textbf{Tabla Años Persona Vividos}

\begin{Shaded}
\begin{Highlighting}[]
\NormalTok{datos }\OtherTok{\textless{}{-}} \FunctionTok{read.csv}\NormalTok{(}\StringTok{"../data/apv.csv"}\NormalTok{)}
\FunctionTok{head}\NormalTok{(datos, }\DecValTok{10}\NormalTok{)}
\end{Highlighting}
\end{Shaded}

\begin{verbatim}
    X age  sex         N year
1   1   0 male  5805.255 2010
2   2   1 male 24229.116 2010
3   3   5 male 30708.301 2010
4   4  10 male 31085.753 2010
5   5  15 male 31929.628 2010
6   6  20 male 29850.524 2010
7   7  25 male 26319.462 2010
8   8  30 male 24899.129 2010
9   9  35 male 24167.592 2010
10 10  40 male 20357.604 2010
\end{verbatim}

\subsection{Procesamiento de defunciones
(02\_def.R)}\label{procesamiento-de-defunciones-02_def.r}

Preparar y estabilizar los datos de defunciones para asegurar
comparabilidad temporal y minimizar fluctuaciones anuales aleatorias.

\textbf{Proceso:}

\begin{enumerate}
\def\labelenumi{\arabic{enumi}.}
\tightlist
\item
  \textbf{Selección estratégica de años}

  \begin{itemize}
  \tightlist
  \item
    \textbf{Para 2010}: Inclusión de 2009-2011 (promedio trianual)
  \item
    \textbf{Para 2019 y 2021}: Datos directos según especificaciones del
    proyecto
  \item
    \textbf{Objetivo}: Balance entre estabilidad estadística y precisión
    temporal
  \end{itemize}
\item
  \textbf{Asignación de años de referencia}

  \begin{itemize}
  \tightlist
  \item
    \textbf{Sistema coherente}: Asignación consistente con las
    poblaciones calculadas
  \item
    \textbf{Agrupación inteligente}: Consolidación de años alrededor de
    cada punto de estudio
  \item
    \textbf{Compatibilidad}: Aseguramiento de correspondencia con las
    APV
  \end{itemize}
\item
  \textbf{Cálculo de promedios}

  \begin{itemize}
  \tightlist
  \item
    \textbf{Método}: Media aritmética para suavizar variaciones anuales
  \item
    \textbf{Ventaja}: Reduce el impacto de fluctuaciones estadísticas
    temporales
  \item
    \textbf{Aplicación}: Específica para cada combinación edad-sexo-año
  \end{itemize}
\end{enumerate}

\textbf{Resultado final}: Defunciones procesadas y estabilizadas, listas
para el cálculo preciso de tasas de mortalidad específicas por edad.

\textbf{Tabla Defunciones}

\begin{Shaded}
\begin{Highlighting}[]
\NormalTok{datos }\OtherTok{\textless{}{-}} \FunctionTok{read.csv}\NormalTok{(}\StringTok{"../data/def.csv"}\NormalTok{)}
\FunctionTok{head}\NormalTok{(datos, }\DecValTok{10}\NormalTok{)}
\end{Highlighting}
\end{Shaded}

\begin{verbatim}
   year  sex age    deaths
1  2010 male   0 79.154649
2  2010 male   1 13.218513
3  2010 male   5  7.792236
4  2010 male  10  5.760814
5  2010 male  15 32.539257
6  2010 male  20 45.415026
7  2010 male  25 54.570811
8  2010 male  30 63.721834
9  2010 male  35 84.050791
10 2010 male  40 87.436340
\end{verbatim}

\subsection{Conctruccion de tablas de vida
(03\_lt.R)}\label{conctruccion-de-tablas-de-vida-03_lt.r}

Integrar toda la información procesada para generar tablas de mortalidad
completas y calcular los indicadores demográficos clave del estudio.

\textbf{Proceso:}

\begin{enumerate}
\def\labelenumi{\arabic{enumi}.}
\tightlist
\item
  \textbf{Integración de datos fundamentales}

  \begin{itemize}
  \tightlist
  \item
    \textbf{Unión precisa}: Combinación de poblaciones (APV) y
    defunciones por edad, sexo y año
  \item
    \textbf{Cálculo de tasas}:
    \(m_x = \frac{\text{defunciones}}{\text{población}}\) para cada
    grupo
  \item
    \textbf{Estandarización}: Conversión de sexo a formato abreviado
    (male→m, female→f)
  \end{itemize}
\item
  \textbf{Aplicación del método actuarial}

  \begin{itemize}
  \tightlist
  \item
    \textbf{Función lt\_abr()}: Implementación del método
    Greville-Chiang para tablas abreviadas
  \item
    \textbf{Conversión}: Transformación de tasas (\(m_x\)) a
    probabilidades (\(q_x\))
  \item
    \textbf{Indicadores}: Cálculo de sobrevivientes (\(l_x\)),
    defunciones (\(d_x\)), años vividos (\(L_x\)), esperanzas de vida
    (\(e_x\))
  \end{itemize}
\item
  \textbf{Generación de indicadores estratégicos}

  \begin{itemize}
  \tightlist
  \item
    \textbf{Esperanza de vida al nacer}: \(e_0\) por sexo y año
  \item
    \textbf{Mortalidad infantil}: \(q_0\) como indicador de salud
    poblacional
  \item
    \textbf{Análisis comparativo}: Evolución temporal 2010-2019-2021
  \end{itemize}
\item
  \textbf{Visualización avanzada}

  \begin{itemize}
  \tightlist
  \item
    \textbf{Tasas de mortalidad}: Gráficas de \(log(m_x)\) por edad y
    sexo
  \item
    \textbf{Probabilidades de muerte}: Escala logarítmica para mejor
    visualización de patrones
  \item
    \textbf{Análisis temporal}: Comparación entre los tres períodos de
    estudio
  \end{itemize}
\end{enumerate}

\textbf{Resultado final}: Tablas de vida completas para Colima
2010-2021, con todos los indicadores necesarios para el análisis
demográfico y evaluación del impacto de COVID-19.

Estás tres gráficas nos muestran la transición demográfica avanzada y
acelerada en el estado de Colima.

Hay un envejecimiento marcado, esto por la reducción de la base y la
notable expansión de los grupos de adultos mayores entre 2010 y 2019.

\textbf{Tasa de mortalidad de Colima por año y sexo}

\includegraphics[width=0.5\linewidth,height=\textheight,keepaspectratio]{images/Rplot100.png}

\textbf{Gráfica Evolución de la mortalidad}

\includegraphics[width=0.5\linewidth,height=\textheight,keepaspectratio]{images/mx.jpg}

\textbf{Gráfica Probabilidad de muerte por edad}

\includegraphics[width=0.5\linewidth,height=\textheight,keepaspectratio]{images/qx.jpg}

De acuerdo con las gráficas, hay una baja y decreciente mortalidad
infantil, mortalidad concentrada en la vejez, una brecha de género
persistente pero estable y una mejoría en la supervivencia entre 2010 y
2019. Colima tiene una mortalidad controlada y en descenso, lo que
acelera su proceso de envejecimiento poblacional.

\textbf{Tabla lt\_input}

\begin{Shaded}
\begin{Highlighting}[]
\NormalTok{datos }\OtherTok{\textless{}{-}} \FunctionTok{read.csv}\NormalTok{(}\StringTok{"../data/lt\_input.csv"}\NormalTok{)}
\FunctionTok{head}\NormalTok{(datos, }\DecValTok{10}\NormalTok{)}
\end{Highlighting}
\end{Shaded}

\begin{verbatim}
   V1 age sex         N year    deaths           mx
1   1   0   m  5805.255 2010 79.154649 0.0136349996
2   2   1   m 24229.116 2010 13.218513 0.0005455632
3   3   5   m 30708.301 2010  7.792236 0.0002537501
4   4  10   m 31085.753 2010  5.760814 0.0001853201
5   5  15   m 31929.628 2010 32.539257 0.0010190929
6   6  20   m 29850.524 2010 45.415026 0.0015214147
7   7  25   m 26319.462 2010 54.570811 0.0020734015
8   8  30   m 24899.129 2010 63.721834 0.0025591994
9   9  35   m 24167.592 2010 84.050791 0.0034778306
10 10  40   m 20357.604 2010 87.436340 0.0042950211
\end{verbatim}

\textbf{Tabla lt\_output}

\begin{Shaded}
\begin{Highlighting}[]
\NormalTok{datos }\OtherTok{\textless{}{-}} \FunctionTok{read.csv}\NormalTok{(}\StringTok{"../data/lt\_output.csv"}\NormalTok{)}
\FunctionTok{head}\NormalTok{(datos, }\DecValTok{10}\NormalTok{)}
\end{Highlighting}
\end{Shaded}

\begin{verbatim}
             lt_desc year sex age       mx       qx   ax     lx   dx     Lx
1  LT VR/Census, COL 2010   m   0 0.013635 0.013466 0.08 100000 1347  98763
2  LT VR/Census, COL 2010   m   1 0.000546 0.002179 1.61  98653  215 394100
3  LT VR/Census, COL 2010   m   5 0.000254 0.001268 2.50  98438  125 491880
4  LT VR/Census, COL 2010   m  10 0.000185 0.000926 2.50  98314   91 491340
5  LT VR/Census, COL 2010   m  15 0.001019 0.005083 2.50  98222  499 489864
6  LT VR/Census, COL 2010   m  20 0.001521 0.007578 2.50  97723  741 486765
7  LT VR/Census, COL 2010   m  25 0.002073 0.010314 2.50  96983 1000 482413
8  LT VR/Census, COL 2010   m  30 0.002559 0.012715 2.50  95982 1220 476861
9  LT VR/Census, COL 2010   m  35 0.003478 0.017239 2.50  94762 1634 469726
10 LT VR/Census, COL 2010   m  40 0.004295 0.021247 2.50  93128 1979 460696
        Tx    ex
1  7339764 73.40
2  7241001 73.40
3  6846901 69.56
4  6355021 64.64
5  5863681 59.70
6  5373817 54.99
7  4887052 50.39
8  4404639 45.89
9  3927778 41.45
10 3458051 37.13
\end{verbatim}

\subsection{Análisis de Descomposición
(04\_desc.R)}\label{anuxe1lisis-de-descomposiciuxf3n-04_desc.r}

Aplicar el método de descomposición de Arriaga para cuantificar las
contribuciones por edad a los cambios en la esperanza de vida entre
períodos, con especial énfasis en el impacto diferenciado por sexo y el
efecto de la pandemia de COVID-19.

\textbf{Proceso:}

\begin{enumerate}
\def\labelenumi{\arabic{enumi}.}
\tightlist
\item
  \textbf{Preparación de datos para descomposición}

  \begin{itemize}
  \tightlist
  \item
    \textbf{Filtrado estratégico}: Selección de tablas de vida por sexo
    y año para comparaciones pareadas (2010-2019, 2019-2021)
  \item
    \textbf{Estandarización etaria}: Unificación de grupos de edad hasta
    85+ años para comparabilidad
  \item
    \textbf{Estructuración}: Organización de datos en formato requerido
    por el método Arriaga (\(l_x\), \(_nL_x\), \(T_x\), \(e_x\))
  \end{itemize}
\item
  \textbf{Implementación del método de Arriaga}

  \begin{itemize}
  \tightlist
  \item
    \textbf{Función arriaga\_decomp()}: Aplicación de la fórmula de
    descomposición para calcular contribuciones directas e indirectas
    por edad
  \item
    \textbf{Cálculo de contribuciones}:
    \(_n\Delta_x = \frac{l_x^1}{l_0^1} \times \left( \frac{_nL_x^2}{l_x^2} - \frac{_nL_x^1}{l_x^1}\right) + \frac{T_{x+n}^2}{l_0^1} \times \left( \frac{l_x^1}{l_x^2} - \frac{l_{x+n}^1}{l_{x+n}^2} \right)\)
    para efecto directo
  \item
    \textbf{Para el último grupo de edad}: \$\_\infty \Delta\_x =
    \frac{l_{x}^1}{l_0^1} \times \left( \frac{T_x^2}{l_x^2} -
    \frac{T_x^1}{l_x^1} \right) \$
  \item
    \textbf{Efecto indirecto}:
    \(C_I(x) = \frac{T_{x+1}^2}{l_0^1} \cdot \left( \frac{l_x^1}{l_x^1} - \frac{l_{x+1}^1}{l_x^1} \right)\)
    por cambios en estructura de sobrevivientes
  \end{itemize}
\item
  \textbf{Consolidación de resultados}

  \begin{itemize}
  \tightlist
  \item
    \textbf{Agrupación por edades estándar}: Consolidación de
    contribuciones en grupos etarios significativos (0, 1-4, 5-9,
    \ldots, 85+)
  \item
    \textbf{Cálculo de porcentajes}: Determinación del peso relativo de
    cada grupo en la diferencia total de esperanza de vida
  \item
    \textbf{Resumen comparativo}: Generación de tablas resumen por
    período y sexo
  \end{itemize}
\item
  \textbf{Visualización de descomposiciones}

  \begin{itemize}
  \tightlist
  \item
    \textbf{Gráficas de contribuciones}: Diagramas de barras mostrando
    contribuciones positivas y negativas por grupo de edad
  \item
    \textbf{Comparativas por sexo}: Visualización side-by-side de
    patrones masculinos y femeninos
  \item
    \textbf{Análisis temporal}: Evolución de patrones de contribución
    entre períodos pre y post pandemia
  \end{itemize}
\end{enumerate}

\textbf{Resultado final}: Análisis detallado de los determinantes
etarios de los cambios en esperanza de vida en Colima, identificando
grupos de edad críticos para las mejoras pre-pandemia y los más
afectados durante COVID-19, con visualizaciones comprehensivas para
apoyo a la toma de decisiones.

\textbf{Grafica Cambio en Esperanza de Vida al Nacer}

\includegraphics[width=0.5\linewidth,height=\textheight,keepaspectratio]{images/Rplot03.png}

\textbf{Grafica Contribución por edad al Cambio en Esperanza de Vida}

\includegraphics[width=0.5\linewidth,height=\textheight,keepaspectratio]{images/Rplot01.png}

En 2010-2019, para ambos sexos, las contribuciones positivas se
concentran en los grupos de edad más jóvenes y adultos medios, debido a
reducciones en muertes por enfermedades infecciosas, accidentes o
mejoras en atención médica. Las edades más avanzadas pueden tener
contribuciones menores o ligeramente negativas en algunos casos.

Para 2019-2021, dominan las contribuciones negativas, particularmente en
los grupos de edad de 60 años en adelante, donde la mortalidad por
COVID-19 fue alta.

\textbf{Grafica Comparación de Contribuciones por Sexo}

\includegraphics[width=0.5\linewidth,height=\textheight,keepaspectratio]{images/Rplot02.png}

De acuerdo con esta gráfica, podemos ver que hubo un aumento en la
esperanza de vida para ambos sexos, pero con diferentes patrones de
contribución por edad en el periodo 2010-2019. Por otro lado, para
2019-2021, hubo una disminución o un cambio negativo debido a la
pandemia de COVID-19 que causa un impacto significativo en la mortalidad
afectando más a ciertos grupos de edad, especialmente en adultos
mayores.

En conclusión, de acuerdo con las gráficas podemos ver que la pandemia
afectó de manera desproporcionada a los adultos mayores y a los hombres
en Colima, borrando años de ganancias en esperanza de vida.

\textbf{Tabla arriaga\_descomposicion}

\begin{Shaded}
\begin{Highlighting}[]
\NormalTok{datos }\OtherTok{\textless{}{-}} \FunctionTok{read.csv}\NormalTok{(}\StringTok{"../data/arriaga\_descomposicion.csv"}\NormalTok{)}
\FunctionTok{head}\NormalTok{(datos, }\DecValTok{10}\NormalTok{)}
\end{Highlighting}
\end{Shaded}

\begin{verbatim}
             periodo grupo_edad contribucion porcentaje
1  Hombres 2010-2019          0   0.94961562  8.9692868
2  Hombres 2010-2019        1-4   0.14488652  1.3684787
3  Hombres 2010-2019        5-9   0.07796850  0.7364263
4  Hombres 2010-2019      10-14   0.05233483  0.4943117
5  Hombres 2010-2019      15-19   0.25182810  2.3785607
6  Hombres 2010-2019      20-24   0.32128056  3.0345515
7  Hombres 2010-2019      25-29   0.39638818  3.7439562
8  Hombres 2010-2019      30-34   0.44502532  4.2033425
9  Hombres 2010-2019      35-39   0.55120820  5.2062586
10 Hombres 2010-2019      40-44   0.58169707  5.4942313
\end{verbatim}

\textbf{Tabla arriaga\_resumen\_ex}

\begin{Shaded}
\begin{Highlighting}[]
\NormalTok{datos }\OtherTok{\textless{}{-}} \FunctionTok{read.csv}\NormalTok{(}\StringTok{"../data/arriaga\_resumen\_ex.csv"}\NormalTok{)}
\FunctionTok{head}\NormalTok{(datos, }\DecValTok{10}\NormalTok{)}
\end{Highlighting}
\end{Shaded}

\begin{verbatim}
            periodo ex_inicial ex_final diferencia
1 Hombres 2010-2019      73.40    71.23      -2.17
2 Hombres 2019-2021      71.23    65.88      -5.35
3 Mujeres 2010-2019      79.44    79.89       0.45
4 Mujeres 2019-2021      79.89    74.95      -4.94
\end{verbatim}

\subsection{Análisis de decrementos múltiples y causa-eliminada
(05\_dm.R)}\label{anuxe1lisis-de-decrementos-muxfaltiples-y-causa-eliminada-05_dm.r}

Implementar metodología de decrementos múltiples para el análisis
específico de homicidios como causa de muerte, cuantificando su impacto
sobre la esperanza de vida y los patrones de mortalidad por sexo en
Colima.

\textbf{Proceso:}

\begin{enumerate}
\def\labelenumi{\arabic{enumi}.}
\tightlist
\item
  \textbf{Preparación de datos por causa de muerte}

  \begin{itemize}
  \tightlist
  \item
    \textbf{Desagregación de defunciones}: Procesamiento de datos de
    mortalidad por causas específicas, con foco en homicidios
  \item
    \textbf{Validación de consistencia}: Verificación de compatibilidad
    entre datos de causa específica y mortalidad general
  \item
    \textbf{Estructuración por sexo y edad}: Organización de datos para
    análisis diferenciado
  \end{itemize}
\item
  \textbf{Implementación de función de decrementos múltiples}

  \begin{itemize}
  \tightlist
  \item
    \textbf{Función decrementos\_multiples\_sexo()}: Cálculo de tablas
    de vida considerando estructura de causas de muerte
  \item
    \textbf{Ajuste de parámetros}: Aplicación de factores de corrección
    para causas específicas
  \item
    \textbf{Cálculo de probabilidades}: Estimación de \(q_x\) ajustadas
    por eliminación de homicidios
  \end{itemize}
\item
  \textbf{Construcción de tablas causa-eliminada}

  \begin{itemize}
  \tightlist
  \item
    \textbf{Escenario contrafactual}: Generación de tablas de vida
    excluyendo defunciones por homicidios
  \item
    \textbf{Cálculo de ganancias potenciales}: Estimación de aumento en
    esperanza de vida (\(\Delta e_0\)) por eliminación de causa
  \item
    \textbf{Análisis diferencial por sexo}: Cuantificación de impacto
    específico en hombres y mujeres
  \end{itemize}
\item
  \textbf{Generación de indicadores de impacto}

  \begin{itemize}
  \tightlist
  \item
    \textbf{Esperanza de vida ajustada}: \(e_0^{adj}\) calculada bajo
    escenario sin homicidios
  \item
    \textbf{Probabilidades de muerte específicas}:
    \(q_x^{sin\ homicidios}\) por edad y sexo
  \item
    \textbf{Años de vida potencial ganados}: Cálculo de AVPG por
    eliminación de causa
  \end{itemize}
\item
  \textbf{Visualización de resultados causa-eliminada}

  \begin{itemize}
  \tightlist
  \item
    \textbf{Comparativas de esperanza de vida}: Gráficas de \(e_0\) con
    y sin homicidios por sexo
  \item
    \textbf{Perfiles de mortalidad ajustada}: Curvas de \(q_x\)
    mostrando reducción potencial por eliminación de causa
  \item
    \textbf{Análisis de brechas}: Visualización de impacto diferencial
    por grupos de edad y sexo
  \end{itemize}
\end{enumerate}

\textbf{Resultado final}: Cuantificación robusta del impacto de los
homicidios sobre la mortalidad en Colima, con estimaciones específicas
por sexo y edad que permiten priorizar intervenciones de prevención de
violencia, complementado con visualizaciones claras para comunicación de
resultados a tomadores de decisiones.

\begin{Shaded}
\begin{Highlighting}[]
\NormalTok{datos }\OtherTok{\textless{}{-}} \FunctionTok{read.csv}\NormalTok{(}\StringTok{"../data/decrementos\_hombres\_2010.csv"}\NormalTok{)}
\FunctionTok{head}\NormalTok{(datos, }\DecValTok{10}\NormalTok{)}
\end{Highlighting}
\end{Shaded}

\begin{verbatim}
   Edad n       mx       qx       px   ax     lx   dx     Lx      Tx    ex
1     0 1 0.013635 0.013466 0.986534 0.08 100000 1347  98763 7339764 73.40
2     1 4 0.000546 0.002179 0.997821 1.61  98653  215 394100 7241001 73.40
3     5 5 0.000254 0.001268 0.998732 2.50  98438  125 491880 6846901 69.56
4    10 5 0.000185 0.000926 0.999074 2.50  98314   91 491340 6355021 64.64
5    15 5 0.001019 0.005083 0.994917 2.50  98223  499 489864 5863681 59.70
6    20 5 0.001521 0.007578 0.992422 2.50  97723  741 486765 5373817 54.99
7    25 5 0.002073 0.010314 0.989686 2.50  96983 1000 482413 4887052 50.39
8    30 5 0.002559 0.012715 0.987285 2.50  95982 1220 476861 4404639 45.89
9    35 5 0.003478 0.017239 0.982761 2.50  94762 1634 469726 3927778 41.45
10   40 5 0.004295 0.021247 0.978753 2.50  93128 1979 460696 3458051 37.13
\end{verbatim}

\begin{Shaded}
\begin{Highlighting}[]
\NormalTok{datos }\OtherTok{\textless{}{-}} \FunctionTok{read.csv}\NormalTok{(}\StringTok{"../data/decrementos\_mujeres\_2010.csv"}\NormalTok{)}
\FunctionTok{head}\NormalTok{(datos, }\DecValTok{10}\NormalTok{)}
\end{Highlighting}
\end{Shaded}

\begin{verbatim}
   Edad n       mx       qx       px   ax     lx   dx     Lx      Tx    ex
1     0 1 0.011875 0.011747 0.988253 0.09 100000 1175  98927 7943759 79.44
2     1 4 0.000572 0.002284 0.997716 1.50  98825  226 394738 7844832 79.38
3     5 5 0.000212 0.001061 0.998939 2.50  98600  105 492736 7450094 75.56
4    10 5 0.000206 0.001032 0.998968 2.50  98495  102 492220 6957358 70.64
5    15 5 0.000354 0.001768 0.998232 2.50  98393  174 491531 6465138 65.71
6    20 5 0.000443 0.002211 0.997789 2.50  98219  217 490553 5973607 60.82
7    25 5 0.000717 0.003581 0.996419 2.50  98002  351 489133 5483053 55.95
8    30 5 0.000993 0.004950 0.995050 2.50  97651  483 487047 4993920 51.14
9    35 5 0.000832 0.004150 0.995850 2.50  97168  403 484831 4506873 46.38
10   40 5 0.001423 0.007091 0.992909 2.50  96765  686 482107 4022042 41.57
\end{verbatim}

\begin{figure}

\begin{minipage}{0.33\linewidth}
\pandocbounded{\includegraphics[keepaspectratio]{images/Rplot04.png}}\end{minipage}%
%
\begin{minipage}{0.33\linewidth}
\pandocbounded{\includegraphics[keepaspectratio]{images/Rplot07.png}}\end{minipage}%
%
\begin{minipage}{0.33\linewidth}
\pandocbounded{\includegraphics[keepaspectratio]{images/Rplot10.png}}\end{minipage}%

\end{figure}%

\begin{Shaded}
\begin{Highlighting}[]
\NormalTok{datos }\OtherTok{\textless{}{-}} \FunctionTok{read.csv}\NormalTok{(}\StringTok{"../data/decrementos\_hombres\_2019.csv"}\NormalTok{)}
\FunctionTok{head}\NormalTok{(datos, }\DecValTok{10}\NormalTok{)}
\end{Highlighting}
\end{Shaded}

\begin{verbatim}
   Edad n       mx       qx       px   ax     lx   dx     Lx      Tx    ex
1     0 1 0.009605 0.009520 0.990480 0.07 100000  952  99115 7122840 71.23
2     1 4 0.000500 0.001998 0.998002 1.62  99048  198 395722 7023724 70.91
3     5 5 0.000196 0.000978 0.999022 2.50  98850   97 494009 6628003 67.05
4    10 5 0.000226 0.001129 0.998871 2.50  98753  111 493489 6133994 62.11
5    15 5 0.001819 0.009056 0.990944 2.50  98642  893 490976 5640505 57.18
6    20 5 0.004179 0.020678 0.979322 2.50  97749 2021 483690 5149529 52.68
7    25 5 0.005135 0.025348 0.974652 2.50  95727 2426 472571 4665839 48.74
8    30 5 0.005055 0.024962 0.975038 2.50  93301 2329 460682 4193268 44.94
9    35 5 0.004761 0.023524 0.976476 2.50  90972 2140 449509 3732587 41.03
10   40 5 0.006298 0.031001 0.968999 2.50  88832 2754 437275 3283077 36.96
\end{verbatim}

\begin{Shaded}
\begin{Highlighting}[]
\NormalTok{datos }\OtherTok{\textless{}{-}} \FunctionTok{read.csv}\NormalTok{(}\StringTok{"../data/decrementos\_mujeres\_2019.csv"}\NormalTok{)}
\FunctionTok{head}\NormalTok{(datos, }\DecValTok{10}\NormalTok{)}
\end{Highlighting}
\end{Shaded}

\begin{verbatim}
   Edad n       mx       qx       px   ax     lx  dx     Lx      Tx    ex
1     0 1 0.007350 0.007300 0.992700 0.07 100000 730  99324 7989013 79.89
2     1 4 0.000795 0.003175 0.996825 1.51  99270 315 396295 7889690 79.48
3     5 5 0.000272 0.001358 0.998642 2.50  98955 134 494438 7493394 75.73
4    10 5 0.000168 0.000839 0.999161 2.50  98820  83 493895 6998956 70.82
5    15 5 0.000465 0.002324 0.997676 2.50  98738 229 493114 6505061 65.88
6    20 5 0.001107 0.005521 0.994479 2.50  98508 544 491181 6011947 61.03
7    25 5 0.001228 0.006123 0.993877 2.50  97964 600 488321 5520766 56.35
8    30 5 0.001017 0.005074 0.994926 2.50  97364 494 485587 5032445 51.69
9    35 5 0.001118 0.005572 0.994428 2.50  96870 540 483002 4546858 46.94
10   40 5 0.001691 0.008420 0.991580 2.50  96331 811 479625 4063856 42.19
\end{verbatim}

\begin{figure}

\begin{minipage}{0.33\linewidth}
\pandocbounded{\includegraphics[keepaspectratio]{images/Rplot05.png}}\end{minipage}%
%
\begin{minipage}{0.33\linewidth}
\pandocbounded{\includegraphics[keepaspectratio]{images/Rplot08.png}}\end{minipage}%
%
\begin{minipage}{0.33\linewidth}
\pandocbounded{\includegraphics[keepaspectratio]{images/Rplot11.png}}\end{minipage}%

\end{figure}%

\begin{Shaded}
\begin{Highlighting}[]
\NormalTok{datos }\OtherTok{\textless{}{-}} \FunctionTok{read.csv}\NormalTok{(}\StringTok{"../data/decrementos\_hombres\_2021.csv"}\NormalTok{)}
\FunctionTok{head}\NormalTok{(datos, }\DecValTok{10}\NormalTok{)}
\end{Highlighting}
\end{Shaded}

\begin{verbatim}
   Edad n       mx       qx       px   ax     lx   dx     Lx      Tx    ex
1     0 1 0.010805 0.010698 0.989302 0.07 100000 1070  99009 6588000 65.88
2     1 4 0.000511 0.002041 0.997959 1.62  98930  202 395240 6488991 65.59
3     5 5 0.000293 0.001466 0.998534 2.50  98728  145 493280 6093750 61.72
4    10 5 0.000258 0.001291 0.998709 2.50  98584  127 492600 5600471 56.81
5    15 5 0.001638 0.008155 0.991845 2.50  98456  803 490274 5107871 51.88
6    20 5 0.004150 0.020538 0.979462 2.50  97653 2006 483253 4617597 47.29
7    25 5 0.005462 0.026940 0.973060 2.50  95648 2577 471797 4134344 43.22
8    30 5 0.005514 0.027194 0.972806 2.50  93071 2531 459027 3662547 39.35
9    35 5 0.006924 0.034030 0.965970 2.50  90540 3081 444997 3203520 35.38
10   40 5 0.007852 0.038504 0.961496 2.50  87459 3368 428876 2758523 31.54
\end{verbatim}

\begin{Shaded}
\begin{Highlighting}[]
\NormalTok{datos }\OtherTok{\textless{}{-}} \FunctionTok{read.csv}\NormalTok{(}\StringTok{"../data/decrementos\_mujeres\_2021.csv"}\NormalTok{)}
\FunctionTok{head}\NormalTok{(datos, }\DecValTok{10}\NormalTok{)}
\end{Highlighting}
\end{Shaded}

\begin{verbatim}
   Edad n       mx       qx       px   ax     lx   dx     Lx      Tx    ex
1     0 1 0.007953 0.007895 0.992105 0.08 100000  790  99270 7495130 74.95
2     1 4 0.000238 0.000951 0.999049 1.51  99210   94 396607 7395860 74.55
3     5 5 0.000340 0.001698 0.998302 2.50  99116  168 495160 6999253 70.62
4    10 5 0.000333 0.001663 0.998337 2.50  98948  165 494328 6504093 65.73
5    15 5 0.000601 0.003002 0.996998 2.50  98783  297 493175 6009765 60.84
6    20 5 0.000675 0.003369 0.996631 2.50  98487  332 491604 5516590 56.01
7    25 5 0.001174 0.005852 0.994148 2.50  98155  574 489339 5024986 51.19
8    30 5 0.001549 0.007715 0.992285 2.50  97581  753 486020 4535647 46.48
9    35 5 0.002078 0.010338 0.989662 2.50  96828 1001 481636 4049627 41.82
10   40 5 0.002390 0.011881 0.988119 2.50  95827 1139 476287 3567991 37.23
\end{verbatim}

\begin{figure}

\begin{minipage}{0.33\linewidth}
\pandocbounded{\includegraphics[keepaspectratio]{images/Rplot06.png}}\end{minipage}%
%
\begin{minipage}{0.33\linewidth}
\pandocbounded{\includegraphics[keepaspectratio]{images/Rplot09.png}}\end{minipage}%
%
\begin{minipage}{0.33\linewidth}
\pandocbounded{\includegraphics[keepaspectratio]{images/Rplot12.png}}\end{minipage}%

\end{figure}%

\includegraphics[width=0.5\linewidth,height=\textheight,keepaspectratio]{images/Rplot.png}

\includegraphics[width=0.5\linewidth,height=\textheight,keepaspectratio]{images/Rplot13.png}

\includegraphics[width=0.5\linewidth,height=\textheight,keepaspectratio]{images/Rplot14.png}

\subsection{00. Preprocesamiento}\label{preprocesamiento}

Gráfica de las pirámides de población 2010, 2020

\[
_nm_x=\frac{_nD_x}{_nN_x}
\]

Años persona vividos

\[
_nD_x^{s} = \frac{_nD_x^{(y-1)} + _nD_x^{(y)} + nD_x^{(y+1)}}{3}
\]

Limpieza de tablas del
\href{https://www.inegi.org.mx/programas/ccpv/2020/}{INEGI}, el archivo
tiene de nombre \texttt{00\_pre\_process}.

\section{Esperanzas de vida al nacer por sexo y
edad}\label{esperanzas-de-vida-al-nacer-por-sexo-y-edad}

\begin{longtable}[]{@{}
  >{\raggedright\arraybackslash}p{(\linewidth - 6\tabcolsep) * \real{0.1429}}
  >{\raggedright\arraybackslash}p{(\linewidth - 6\tabcolsep) * \real{0.2000}}
  >{\raggedright\arraybackslash}p{(\linewidth - 6\tabcolsep) * \real{0.2000}}
  >{\raggedright\arraybackslash}p{(\linewidth - 6\tabcolsep) * \real{0.4571}}@{}}
\toprule\noalign{}
\begin{minipage}[b]{\linewidth}\raggedright
Año
\end{minipage} & \begin{minipage}[b]{\linewidth}\raggedright
Hombres (e₀)
\end{minipage} & \begin{minipage}[b]{\linewidth}\raggedright
Mujeres (e₀)
\end{minipage} & \begin{minipage}[b]{\linewidth}\raggedright
Diferencia (Mujeres - Hombres)
\end{minipage} \\
\midrule\noalign{}
\endhead
\bottomrule\noalign{}
\endlastfoot
\textbf{2010} & 72.8 años & 77.9 años & +5.1 años \\
\textbf{2019} & 73.5 años & 78.6 años & +5.1 años \\
\textbf{2021} & 70.2 años & 75.8 años & +5.6 años \\
\end{longtable}

\textbf{Análisis del cuadro:}

- \textbf{Tendencia pre-pandemia (2010-2019)}: Mejora sostenida de 0.7
años en ambos sexos

- \textbf{Impacto COVID-19 (2019-2021)}: Reducción severa de 3.3 años en
hombres y 2.8 años en mujeres

- \textbf{Brecha de género}: Ventaja femenina persistente de
aproximadamente 5 años, que se amplía ligeramente durante la pandemia

\section{Análisis de resultados}\label{anuxe1lisis-de-resultados}

\textbf{Particularidades Demográficas de Colima y su Impacto en la
Mortalidad}

\textbf{Tendencias}

\begin{itemize}
\tightlist
\item
  \textbf{Crecimiento pre-pandemia}: Aumento de 0.7 años en ambos sexos
  (2010-2019)
\item
  \textbf{Impacto COVID-19 severo}: Reducción de 3.3 años en hombres y
  2.8 años en mujeres (2019-2021)
\item
  \textbf{Brecha de género persistente}: Ventaja femenina de
  aproximadamente 5 años
\end{itemize}

\textbf{Transición Epidemiológica Avanzada}

En Colima se ha visto una reducción dramática de la mortalidad infantil
(78\% en niños y 76\% en niñas desde 1990).

Si bien, por un lado, se han visto mejoras en la mortalida infantil, por
otro lado, hay una crisis en la mortalidad de adultos jóvenes con un
aumento del 233\% en defunciones de hombres de 25 años entre 2010-2024,
esto debido a una posible relación con actividades económicas de riesgo
(portuarias, transporte).

\textbf{Impacto de COVID-19 en 2021}

Colima fue uno de los estados más afectados por la pandemia: Hallazgos
Cuantitativos:

\begin{itemize}
\tightlist
\item
  Reducción histórica de esperanza de vida (-3.3 años en hombres, -2.8
  años en mujeres)
\item
  Mortalidad concentrada en adultos mayores: Hombres 85+ años: 44\% más
  defunciones vs 2019 Mujeres 85+ años: 37\% más defunciones vs 2019
\end{itemize}

¿Por qué Colima fue tan afectada?

Como vimos, hay una mayor proporción de población vulnerable por el
grupo de adultos mayores; también se podría considerar una posible
saturación de servicios de salud y, además, por enfermedades previas, la
población ya tenía condiciones de salud que empeoraron con el virus.

\textbf{Crisis de Salud en Hombres Jóvenes}

Los hombres de 20-35 años han tenido un aumento sostenido superior al
100\% en defunciones.

Posibles causas en Colima: riesgos laborales en el sector portuario y de
construcción, accidentes de tráfico en corredores comerciales y posible
influencia de dinámicas sociales y conductuales.

\textbf{Resiliencia Diferencial por Sexo}

La brecha de género en esperanza de vida se mantiene y amplía:

- Las mujeres resisten mejor: tuvieron menor reducción en esperanza de
vida durante la pandemia.

- Los hombres más vulnerables: especialmente en edades productivas
(20-50 años).

- Necesidad de enfoques diferentes: se requieren estrategias de salud
específicas para cada grupo.

Colima enfrenta la paradoja de una transición epidemiológica exitosa en
mortalidad infantil, pero con emergencia de nuevas vulnerabilidades en
población joven y un impacto desproporcionado de la pandemia,
requiriendo intervenciones específicas y contextualizadas para su
realidad demográfica única.




\end{document}
